\documentclass[12pt, letterpaper]{article}
\usepackage{graphicx}
\usepackage{amsmath}

\title{bimodal analysis}
\author{Brandon Chen}

\begin{document}
\maketitle

\section{Modal}

Bimodal: application scans two arrays simultaneously.

\begin{itemize}
\item $x_1$: size of array 1
\item $x_2$: size of array 2
\item $p$: probability of access array 1
\item $1-p$: probability of access array 2
\item $d_1 = \frac{x_1}{p}$: reuse distance for data in array 1
\item $d_2 = \frac{x_2}{1-p}$: reuse distance for data in array 2
\item $ed = p d_1 + (1-p) d_2$: expected reuse distance, equals the size of
working set 
\item S: cache size
\item m: miss rate
\end{itemize}

The relation between cache size and miss rate:
\begin{equation}
S = \sum a [H(a) + E(a)]
\end{equation}

\section{Policy Analysis}

\subsection{evict at the age of 0}
\begin{equation}
S = (1-m) \times ed
\end{equation}

\subsection{evict at the age of $d_1$}

\begin{equation}
\begin{gathered}
S = m \times d_1 + p \times d_1 + (1-p-m) \times d_2 \\
m = \frac{ed - S}{d_2-d_1}
\end{gathered}
\end{equation}

\subsection{first evict at 0, then evict d1}
First evict at age 0, then evict $d_1$ if there appears. If no d1 candidates,
keep evicting 0. So hits only happen at $d_1$.

Here we use $x$ to denote candidates who make it to the age of d1. Hits only
happen at age $d_1$ means: $px = 1-m$.
\begin{equation}
\begin{gathered}
S = (1-m) d_1 + (1-p) x d_1 \\
m = 1 - \frac{pS}{d_1}
\end{gathered}
\end{equation}

\end{document}
