\documentclass[12pt, letterpaper]{article}
\usepackage{graphicx}
\usepackage{amsmath}
\graphicspath{ {../graphs/} }

\title{The study of eviction age distribution in bimodal distribution with
small cache size}
\author{Brandon}

\begin{document}

\maketitle

\begin{figure}[h]
\includegraphics[width=\textwidth]{hit-evict-2-33}
\caption{hit age and evicion age distritbution from cache size of 2 to 33}
\end{figure}

Observation: Look at the eviction age distributions, where the fraction of
evictions at the age of $D_{1}$ increases as the cache size increases. 

The way I understand this trend is that for a cache smaller than $D_{1}$, it
has to wait for the candidates to grow to the age of $D_1$ so that they can be
hitted or, if not hitted, to be realized it's actually cacheing the big array
and should be evicted. Before that, given our ranking function, cache will
always evict newly added candidates (age $0$) in order to conserve the rest of
the candidates to let them grow as old as $D_{1}$. So how long does it take for
the candidates to grow old enough? It depends on the cache size.

Let's start with an empty cache with the size of S, and with S times of
compulsory misses it is loaded with candidates age from 0 to S-1. Then it takes
$D_{1}-S$ accesses for the oldest candidate to grow to the age of $D_{1}$. And
we have candidates aging from $(D_{1}-S+1,D_{1}-S+2,...,D_{1})$ and a newly
added candidate (age 0). Since then candidates will be hitted at the age of $D_{1}$ with probability of p, and evicted at the age of $D_{1}$ with probability 1-p.
This will repeat until we run out of candidates at age $D_{1}$, which takes S times of accesses. Then we go back to the situation where cache consists of candidates aging from 0 to S-1. So the cycle is $D_{1}$ accesses.

The above analysis implies the fraction of evictions at age $D_{1}$ among all the evictions is the following. Here I use $x$ to denote the this fraction. 


\begin{equation}
\label{eq:fraction}
\begin{split}
x = \frac{pS}{(D_{1}-S)+pS} \\
= \frac{pS}{D_{1}-(1-p)S}
\end{split}
\end{equation}

And equation \ref{eq:fraction} holds true when compared to the simulation
results.

But then the Cache Size equation goes something really weird...

\begin{equation}
\begin{split}
S = (1-m) \times D_{1} + xm \times D_{1} + (1-x)m \times 0 \\
= (1-m)D_{1} + \frac{pS}{D_{1}-(1-p)S} m D_{1} \\
\end{split}
\end{equation}

And then the relationship between miss rate and cache size becomes

\begin{equation}
\begin{split}
m = \frac{(1-p)S^2-(2-p)D_{1}S+{D_{1}}^2}{{D_{1}}^2-D_{1}S}
\end{split}
\end{equation}

\begin{figure}[h]
\includegraphics[width=\textwidth]{access}
\caption{Access pattern, with p=0.5, S1=16, S2=48}
\label{fig:access}
\end{figure}

I did some simulations. This time I made the access parttern less random (
shown in figure \ref{fig:access}). And the model, although very complicated,
looks identical to a straight line.

\begin{figure}[h]
\includegraphics[width=\textwidth]{miss-rate-0-33}
\caption{new model v.s. simulation}
\end{figure}

I thought the almost consistent gap between the model and simulation is caused
by compulsory misses. However, after I do 10 times more iterations (100k),
there is no significant decrease in the gap.

\begin{figure}[h]
\includegraphics[width=\textwidth]{miss-rate-0-33-100k}
\caption{new model v.s. simulation, with 100k times iterations}
\end{figure}

(Note: I got something pretty complicated from terribly simple assumptions. I
was probably wasting time, but I don't understand the gap so I keep this note
anyway.)

\end{document}
