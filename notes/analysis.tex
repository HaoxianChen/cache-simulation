\documentclass[12pt, letterpaper]{article}
\usepackage{graphicx}
\usepackage{amsmath}

\title{bimodal analysis}
\author{Brandon Chen}

\begin{document}
\maketitle

\section{Bimodal}

Bimodal: application scans two arrays simultaneously.

\begin{itemize}
\item $x_1$: size of array 1
\item $x_2$: size of array 2
\item $p$: probability of access array 1
\item $1-p$: probability of access array 2
\item $d_1 = \frac{x_1}{p}$: reuse distance for data in array 1
\item $d_2 = \frac{x_2}{1-p}$: reuse distance for data in array 2
\item $ed = p d_1 + (1-p) d_2$: expected reuse distance, equals the size of
working set 
\item S: cache size
\item m: miss rate
\end{itemize}

The relation between cache size and miss rate:
\begin{equation}
S = \sum a [H(a) + E(a)]
\end{equation}

\subsection{Policy Analysis}

\subsubsection{evict at the age of 0}
\begin{equation}
S = (1-m) \times ed
\end{equation}

\subsubsection{evict at the age of $d_1$}

\begin{equation}
\begin{gathered}
S = m \times d_1 + p \times d_1 + (1-p-m) \times d_2 \\
m = \frac{ed - S}{d_2-d_1}
\end{gathered}
\end{equation}

\subsubsection{first evict at 0, then evict d1}
First evict at age 0, then evict $d_1$ if there appears. If no d1 candidates,
keep evicting 0. So hits only happen at $d_1$.

Here we use $x$ to denote candidates who make it to the age of d1. Hits only
happen at age $d_1$ means: $px = 1-m$.
\begin{equation}
\begin{gathered}
S = (1-m) d_1 + (1-p) x d_1 \\
m = 1 - \frac{pS}{d_1}
\end{gathered}
\end{equation}

\section{Trimodal}

Trimodal: application scans three arrays simultaneously.

\begin{itemize}
\item $x_1$: size of array 1
\item $x_2$: size of array 2
\item $x_3$: size of array 3
\item $p_1$: probability of access array 1
\item $p_2$: probability of access array 2
\item $p_3$: probability of access array 3
\item $d_1 = \frac{x_1}{p_1}$: reuse distance for data in array 1
\item $d_2 = \frac{x_2}{p_2}$: reuse distance for data in array 2
\item $d_3 = \frac{x_3}{p_3}$: reuse distance for data in array 3
\item $ed = p d_1 + p_2 d_2 + p_3 d_3 $: expected reuse distance, equals the
size of working set 
\item S: cache size
\item m: miss rate
\end{itemize}

The relation between cache size and miss rate:
\begin{equation}
S = \sum a [H(a) + E(a)]
\end{equation}

\subsection{Policy Analysis}

\subsubsection{Evict at d1, then 0}
\label{sec:d1-0}

Let x denotes the fraction of candidates that make it to the age of $d_1$.

When $0<S<d_1$, hits only happen at $d_1$. 

\begin{center}
\begin{tabular}{ c | c c c c }
\hline
 		& d1  & d2 & d3\\ 
\hline
 hit 	& 1-m & &\\  
 evict 	& $(1-p_1)x$   & &\\
\hline
\end{tabular}
\end{center}

\begin{equation}
\begin{gathered}
S = (1-p_1)xd_1 + p_1 x d_1 \\
p_1 x = 1 - m \\
m = 1 - \frac{p_1 S}{d_1}
\end{gathered}
\end{equation}

Confirmed by simulation.

When $d_1<S< x_1 + x_2$, the policy can always evict at $d_1$, and either hit at $d_1$
or $d_2$. 
\begin{center}
\begin{tabular}{ c | c c c c }
\hline
      & d1    & d2 & d3\\ 
\hline 
hit   & $p_1$ & $(1-m-p_1)$ &\\
evict & m   & & \\
\hline 
\end{tabular}
\end{center}

\begin{equation}
m = \frac{p_1 d_1 + (1-p_1) d_2 - S}{d_2 - d_1}
\end{equation}

When $ x_1 + x_2 < S $ 

\subsubsection{Evict at d2, then d1, then 0}

When $0<S<d_1$, policy will do the same thing as described in
section~\ref{sec:d1-0}. 

Confirmed by simulation.

When $d_1<S<d<2$, will always end up having candidates at $d_1$?

\subsubsection{Evict at d2, then 0}
\subsubsection{Evict at d1, then d2, then 0}

\end{document}
